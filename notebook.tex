
% Default to the notebook output style

    


% Inherit from the specified cell style.




    
\documentclass[11pt]{article}

    
    
    \usepackage[T1]{fontenc}
    % Nicer default font (+ math font) than Computer Modern for most use cases
    \usepackage{mathpazo}

    % Basic figure setup, for now with no caption control since it's done
    % automatically by Pandoc (which extracts ![](path) syntax from Markdown).
    \usepackage{graphicx}
    % We will generate all images so they have a width \maxwidth. This means
    % that they will get their normal width if they fit onto the page, but
    % are scaled down if they would overflow the margins.
    \makeatletter
    \def\maxwidth{\ifdim\Gin@nat@width>\linewidth\linewidth
    \else\Gin@nat@width\fi}
    \makeatother
    \let\Oldincludegraphics\includegraphics
    % Set max figure width to be 80% of text width, for now hardcoded.
    \renewcommand{\includegraphics}[1]{\Oldincludegraphics[width=.8\maxwidth]{#1}}
    % Ensure that by default, figures have no caption (until we provide a
    % proper Figure object with a Caption API and a way to capture that
    % in the conversion process - todo).
    \usepackage{caption}
    \DeclareCaptionLabelFormat{nolabel}{}
    \captionsetup{labelformat=nolabel}

    \usepackage{adjustbox} % Used to constrain images to a maximum size 
    \usepackage{xcolor} % Allow colors to be defined
    \usepackage{enumerate} % Needed for markdown enumerations to work
    \usepackage{geometry} % Used to adjust the document margins
    \usepackage{amsmath} % Equations
    \usepackage{amssymb} % Equations
    \usepackage{textcomp} % defines textquotesingle
    % Hack from http://tex.stackexchange.com/a/47451/13684:
    \AtBeginDocument{%
        \def\PYZsq{\textquotesingle}% Upright quotes in Pygmentized code
    }
    \usepackage{upquote} % Upright quotes for verbatim code
    \usepackage{eurosym} % defines \euro
    \usepackage[mathletters]{ucs} % Extended unicode (utf-8) support
    \usepackage[utf8x]{inputenc} % Allow utf-8 characters in the tex document
    \usepackage{fancyvrb} % verbatim replacement that allows latex
    \usepackage{grffile} % extends the file name processing of package graphics 
                         % to support a larger range 
    % The hyperref package gives us a pdf with properly built
    % internal navigation ('pdf bookmarks' for the table of contents,
    % internal cross-reference links, web links for URLs, etc.)
    \usepackage{hyperref}
    \usepackage{longtable} % longtable support required by pandoc >1.10
    \usepackage{booktabs}  % table support for pandoc > 1.12.2
    \usepackage[inline]{enumitem} % IRkernel/repr support (it uses the enumerate* environment)
    \usepackage[normalem]{ulem} % ulem is needed to support strikethroughs (\sout)
                                % normalem makes italics be italics, not underlines
    

    
    
    % Colors for the hyperref package
    \definecolor{urlcolor}{rgb}{0,.145,.698}
    \definecolor{linkcolor}{rgb}{.71,0.21,0.01}
    \definecolor{citecolor}{rgb}{.12,.54,.11}

    % ANSI colors
    \definecolor{ansi-black}{HTML}{3E424D}
    \definecolor{ansi-black-intense}{HTML}{282C36}
    \definecolor{ansi-red}{HTML}{E75C58}
    \definecolor{ansi-red-intense}{HTML}{B22B31}
    \definecolor{ansi-green}{HTML}{00A250}
    \definecolor{ansi-green-intense}{HTML}{007427}
    \definecolor{ansi-yellow}{HTML}{DDB62B}
    \definecolor{ansi-yellow-intense}{HTML}{B27D12}
    \definecolor{ansi-blue}{HTML}{208FFB}
    \definecolor{ansi-blue-intense}{HTML}{0065CA}
    \definecolor{ansi-magenta}{HTML}{D160C4}
    \definecolor{ansi-magenta-intense}{HTML}{A03196}
    \definecolor{ansi-cyan}{HTML}{60C6C8}
    \definecolor{ansi-cyan-intense}{HTML}{258F8F}
    \definecolor{ansi-white}{HTML}{C5C1B4}
    \definecolor{ansi-white-intense}{HTML}{A1A6B2}

    % commands and environments needed by pandoc snippets
    % extracted from the output of `pandoc -s`
    \providecommand{\tightlist}{%
      \setlength{\itemsep}{0pt}\setlength{\parskip}{0pt}}
    \DefineVerbatimEnvironment{Highlighting}{Verbatim}{commandchars=\\\{\}}
    % Add ',fontsize=\small' for more characters per line
    \newenvironment{Shaded}{}{}
    \newcommand{\KeywordTok}[1]{\textcolor[rgb]{0.00,0.44,0.13}{\textbf{{#1}}}}
    \newcommand{\DataTypeTok}[1]{\textcolor[rgb]{0.56,0.13,0.00}{{#1}}}
    \newcommand{\DecValTok}[1]{\textcolor[rgb]{0.25,0.63,0.44}{{#1}}}
    \newcommand{\BaseNTok}[1]{\textcolor[rgb]{0.25,0.63,0.44}{{#1}}}
    \newcommand{\FloatTok}[1]{\textcolor[rgb]{0.25,0.63,0.44}{{#1}}}
    \newcommand{\CharTok}[1]{\textcolor[rgb]{0.25,0.44,0.63}{{#1}}}
    \newcommand{\StringTok}[1]{\textcolor[rgb]{0.25,0.44,0.63}{{#1}}}
    \newcommand{\CommentTok}[1]{\textcolor[rgb]{0.38,0.63,0.69}{\textit{{#1}}}}
    \newcommand{\OtherTok}[1]{\textcolor[rgb]{0.00,0.44,0.13}{{#1}}}
    \newcommand{\AlertTok}[1]{\textcolor[rgb]{1.00,0.00,0.00}{\textbf{{#1}}}}
    \newcommand{\FunctionTok}[1]{\textcolor[rgb]{0.02,0.16,0.49}{{#1}}}
    \newcommand{\RegionMarkerTok}[1]{{#1}}
    \newcommand{\ErrorTok}[1]{\textcolor[rgb]{1.00,0.00,0.00}{\textbf{{#1}}}}
    \newcommand{\NormalTok}[1]{{#1}}
    
    % Additional commands for more recent versions of Pandoc
    \newcommand{\ConstantTok}[1]{\textcolor[rgb]{0.53,0.00,0.00}{{#1}}}
    \newcommand{\SpecialCharTok}[1]{\textcolor[rgb]{0.25,0.44,0.63}{{#1}}}
    \newcommand{\VerbatimStringTok}[1]{\textcolor[rgb]{0.25,0.44,0.63}{{#1}}}
    \newcommand{\SpecialStringTok}[1]{\textcolor[rgb]{0.73,0.40,0.53}{{#1}}}
    \newcommand{\ImportTok}[1]{{#1}}
    \newcommand{\DocumentationTok}[1]{\textcolor[rgb]{0.73,0.13,0.13}{\textit{{#1}}}}
    \newcommand{\AnnotationTok}[1]{\textcolor[rgb]{0.38,0.63,0.69}{\textbf{\textit{{#1}}}}}
    \newcommand{\CommentVarTok}[1]{\textcolor[rgb]{0.38,0.63,0.69}{\textbf{\textit{{#1}}}}}
    \newcommand{\VariableTok}[1]{\textcolor[rgb]{0.10,0.09,0.49}{{#1}}}
    \newcommand{\ControlFlowTok}[1]{\textcolor[rgb]{0.00,0.44,0.13}{\textbf{{#1}}}}
    \newcommand{\OperatorTok}[1]{\textcolor[rgb]{0.40,0.40,0.40}{{#1}}}
    \newcommand{\BuiltInTok}[1]{{#1}}
    \newcommand{\ExtensionTok}[1]{{#1}}
    \newcommand{\PreprocessorTok}[1]{\textcolor[rgb]{0.74,0.48,0.00}{{#1}}}
    \newcommand{\AttributeTok}[1]{\textcolor[rgb]{0.49,0.56,0.16}{{#1}}}
    \newcommand{\InformationTok}[1]{\textcolor[rgb]{0.38,0.63,0.69}{\textbf{\textit{{#1}}}}}
    \newcommand{\WarningTok}[1]{\textcolor[rgb]{0.38,0.63,0.69}{\textbf{\textit{{#1}}}}}
    
    
    % Define a nice break command that doesn't care if a line doesn't already
    % exist.
    \def\br{\hspace*{\fill} \\* }
    % Math Jax compatability definitions
    \def\gt{>}
    \def\lt{<}
    % Document parameters
    \title{Untitled}
    
    
    

    % Pygments definitions
    
\makeatletter
\def\PY@reset{\let\PY@it=\relax \let\PY@bf=\relax%
    \let\PY@ul=\relax \let\PY@tc=\relax%
    \let\PY@bc=\relax \let\PY@ff=\relax}
\def\PY@tok#1{\csname PY@tok@#1\endcsname}
\def\PY@toks#1+{\ifx\relax#1\empty\else%
    \PY@tok{#1}\expandafter\PY@toks\fi}
\def\PY@do#1{\PY@bc{\PY@tc{\PY@ul{%
    \PY@it{\PY@bf{\PY@ff{#1}}}}}}}
\def\PY#1#2{\PY@reset\PY@toks#1+\relax+\PY@do{#2}}

\expandafter\def\csname PY@tok@w\endcsname{\def\PY@tc##1{\textcolor[rgb]{0.73,0.73,0.73}{##1}}}
\expandafter\def\csname PY@tok@c\endcsname{\let\PY@it=\textit\def\PY@tc##1{\textcolor[rgb]{0.25,0.50,0.50}{##1}}}
\expandafter\def\csname PY@tok@cp\endcsname{\def\PY@tc##1{\textcolor[rgb]{0.74,0.48,0.00}{##1}}}
\expandafter\def\csname PY@tok@k\endcsname{\let\PY@bf=\textbf\def\PY@tc##1{\textcolor[rgb]{0.00,0.50,0.00}{##1}}}
\expandafter\def\csname PY@tok@kp\endcsname{\def\PY@tc##1{\textcolor[rgb]{0.00,0.50,0.00}{##1}}}
\expandafter\def\csname PY@tok@kt\endcsname{\def\PY@tc##1{\textcolor[rgb]{0.69,0.00,0.25}{##1}}}
\expandafter\def\csname PY@tok@o\endcsname{\def\PY@tc##1{\textcolor[rgb]{0.40,0.40,0.40}{##1}}}
\expandafter\def\csname PY@tok@ow\endcsname{\let\PY@bf=\textbf\def\PY@tc##1{\textcolor[rgb]{0.67,0.13,1.00}{##1}}}
\expandafter\def\csname PY@tok@nb\endcsname{\def\PY@tc##1{\textcolor[rgb]{0.00,0.50,0.00}{##1}}}
\expandafter\def\csname PY@tok@nf\endcsname{\def\PY@tc##1{\textcolor[rgb]{0.00,0.00,1.00}{##1}}}
\expandafter\def\csname PY@tok@nc\endcsname{\let\PY@bf=\textbf\def\PY@tc##1{\textcolor[rgb]{0.00,0.00,1.00}{##1}}}
\expandafter\def\csname PY@tok@nn\endcsname{\let\PY@bf=\textbf\def\PY@tc##1{\textcolor[rgb]{0.00,0.00,1.00}{##1}}}
\expandafter\def\csname PY@tok@ne\endcsname{\let\PY@bf=\textbf\def\PY@tc##1{\textcolor[rgb]{0.82,0.25,0.23}{##1}}}
\expandafter\def\csname PY@tok@nv\endcsname{\def\PY@tc##1{\textcolor[rgb]{0.10,0.09,0.49}{##1}}}
\expandafter\def\csname PY@tok@no\endcsname{\def\PY@tc##1{\textcolor[rgb]{0.53,0.00,0.00}{##1}}}
\expandafter\def\csname PY@tok@nl\endcsname{\def\PY@tc##1{\textcolor[rgb]{0.63,0.63,0.00}{##1}}}
\expandafter\def\csname PY@tok@ni\endcsname{\let\PY@bf=\textbf\def\PY@tc##1{\textcolor[rgb]{0.60,0.60,0.60}{##1}}}
\expandafter\def\csname PY@tok@na\endcsname{\def\PY@tc##1{\textcolor[rgb]{0.49,0.56,0.16}{##1}}}
\expandafter\def\csname PY@tok@nt\endcsname{\let\PY@bf=\textbf\def\PY@tc##1{\textcolor[rgb]{0.00,0.50,0.00}{##1}}}
\expandafter\def\csname PY@tok@nd\endcsname{\def\PY@tc##1{\textcolor[rgb]{0.67,0.13,1.00}{##1}}}
\expandafter\def\csname PY@tok@s\endcsname{\def\PY@tc##1{\textcolor[rgb]{0.73,0.13,0.13}{##1}}}
\expandafter\def\csname PY@tok@sd\endcsname{\let\PY@it=\textit\def\PY@tc##1{\textcolor[rgb]{0.73,0.13,0.13}{##1}}}
\expandafter\def\csname PY@tok@si\endcsname{\let\PY@bf=\textbf\def\PY@tc##1{\textcolor[rgb]{0.73,0.40,0.53}{##1}}}
\expandafter\def\csname PY@tok@se\endcsname{\let\PY@bf=\textbf\def\PY@tc##1{\textcolor[rgb]{0.73,0.40,0.13}{##1}}}
\expandafter\def\csname PY@tok@sr\endcsname{\def\PY@tc##1{\textcolor[rgb]{0.73,0.40,0.53}{##1}}}
\expandafter\def\csname PY@tok@ss\endcsname{\def\PY@tc##1{\textcolor[rgb]{0.10,0.09,0.49}{##1}}}
\expandafter\def\csname PY@tok@sx\endcsname{\def\PY@tc##1{\textcolor[rgb]{0.00,0.50,0.00}{##1}}}
\expandafter\def\csname PY@tok@m\endcsname{\def\PY@tc##1{\textcolor[rgb]{0.40,0.40,0.40}{##1}}}
\expandafter\def\csname PY@tok@gh\endcsname{\let\PY@bf=\textbf\def\PY@tc##1{\textcolor[rgb]{0.00,0.00,0.50}{##1}}}
\expandafter\def\csname PY@tok@gu\endcsname{\let\PY@bf=\textbf\def\PY@tc##1{\textcolor[rgb]{0.50,0.00,0.50}{##1}}}
\expandafter\def\csname PY@tok@gd\endcsname{\def\PY@tc##1{\textcolor[rgb]{0.63,0.00,0.00}{##1}}}
\expandafter\def\csname PY@tok@gi\endcsname{\def\PY@tc##1{\textcolor[rgb]{0.00,0.63,0.00}{##1}}}
\expandafter\def\csname PY@tok@gr\endcsname{\def\PY@tc##1{\textcolor[rgb]{1.00,0.00,0.00}{##1}}}
\expandafter\def\csname PY@tok@ge\endcsname{\let\PY@it=\textit}
\expandafter\def\csname PY@tok@gs\endcsname{\let\PY@bf=\textbf}
\expandafter\def\csname PY@tok@gp\endcsname{\let\PY@bf=\textbf\def\PY@tc##1{\textcolor[rgb]{0.00,0.00,0.50}{##1}}}
\expandafter\def\csname PY@tok@go\endcsname{\def\PY@tc##1{\textcolor[rgb]{0.53,0.53,0.53}{##1}}}
\expandafter\def\csname PY@tok@gt\endcsname{\def\PY@tc##1{\textcolor[rgb]{0.00,0.27,0.87}{##1}}}
\expandafter\def\csname PY@tok@err\endcsname{\def\PY@bc##1{\setlength{\fboxsep}{0pt}\fcolorbox[rgb]{1.00,0.00,0.00}{1,1,1}{\strut ##1}}}
\expandafter\def\csname PY@tok@kc\endcsname{\let\PY@bf=\textbf\def\PY@tc##1{\textcolor[rgb]{0.00,0.50,0.00}{##1}}}
\expandafter\def\csname PY@tok@kd\endcsname{\let\PY@bf=\textbf\def\PY@tc##1{\textcolor[rgb]{0.00,0.50,0.00}{##1}}}
\expandafter\def\csname PY@tok@kn\endcsname{\let\PY@bf=\textbf\def\PY@tc##1{\textcolor[rgb]{0.00,0.50,0.00}{##1}}}
\expandafter\def\csname PY@tok@kr\endcsname{\let\PY@bf=\textbf\def\PY@tc##1{\textcolor[rgb]{0.00,0.50,0.00}{##1}}}
\expandafter\def\csname PY@tok@bp\endcsname{\def\PY@tc##1{\textcolor[rgb]{0.00,0.50,0.00}{##1}}}
\expandafter\def\csname PY@tok@fm\endcsname{\def\PY@tc##1{\textcolor[rgb]{0.00,0.00,1.00}{##1}}}
\expandafter\def\csname PY@tok@vc\endcsname{\def\PY@tc##1{\textcolor[rgb]{0.10,0.09,0.49}{##1}}}
\expandafter\def\csname PY@tok@vg\endcsname{\def\PY@tc##1{\textcolor[rgb]{0.10,0.09,0.49}{##1}}}
\expandafter\def\csname PY@tok@vi\endcsname{\def\PY@tc##1{\textcolor[rgb]{0.10,0.09,0.49}{##1}}}
\expandafter\def\csname PY@tok@vm\endcsname{\def\PY@tc##1{\textcolor[rgb]{0.10,0.09,0.49}{##1}}}
\expandafter\def\csname PY@tok@sa\endcsname{\def\PY@tc##1{\textcolor[rgb]{0.73,0.13,0.13}{##1}}}
\expandafter\def\csname PY@tok@sb\endcsname{\def\PY@tc##1{\textcolor[rgb]{0.73,0.13,0.13}{##1}}}
\expandafter\def\csname PY@tok@sc\endcsname{\def\PY@tc##1{\textcolor[rgb]{0.73,0.13,0.13}{##1}}}
\expandafter\def\csname PY@tok@dl\endcsname{\def\PY@tc##1{\textcolor[rgb]{0.73,0.13,0.13}{##1}}}
\expandafter\def\csname PY@tok@s2\endcsname{\def\PY@tc##1{\textcolor[rgb]{0.73,0.13,0.13}{##1}}}
\expandafter\def\csname PY@tok@sh\endcsname{\def\PY@tc##1{\textcolor[rgb]{0.73,0.13,0.13}{##1}}}
\expandafter\def\csname PY@tok@s1\endcsname{\def\PY@tc##1{\textcolor[rgb]{0.73,0.13,0.13}{##1}}}
\expandafter\def\csname PY@tok@mb\endcsname{\def\PY@tc##1{\textcolor[rgb]{0.40,0.40,0.40}{##1}}}
\expandafter\def\csname PY@tok@mf\endcsname{\def\PY@tc##1{\textcolor[rgb]{0.40,0.40,0.40}{##1}}}
\expandafter\def\csname PY@tok@mh\endcsname{\def\PY@tc##1{\textcolor[rgb]{0.40,0.40,0.40}{##1}}}
\expandafter\def\csname PY@tok@mi\endcsname{\def\PY@tc##1{\textcolor[rgb]{0.40,0.40,0.40}{##1}}}
\expandafter\def\csname PY@tok@il\endcsname{\def\PY@tc##1{\textcolor[rgb]{0.40,0.40,0.40}{##1}}}
\expandafter\def\csname PY@tok@mo\endcsname{\def\PY@tc##1{\textcolor[rgb]{0.40,0.40,0.40}{##1}}}
\expandafter\def\csname PY@tok@ch\endcsname{\let\PY@it=\textit\def\PY@tc##1{\textcolor[rgb]{0.25,0.50,0.50}{##1}}}
\expandafter\def\csname PY@tok@cm\endcsname{\let\PY@it=\textit\def\PY@tc##1{\textcolor[rgb]{0.25,0.50,0.50}{##1}}}
\expandafter\def\csname PY@tok@cpf\endcsname{\let\PY@it=\textit\def\PY@tc##1{\textcolor[rgb]{0.25,0.50,0.50}{##1}}}
\expandafter\def\csname PY@tok@c1\endcsname{\let\PY@it=\textit\def\PY@tc##1{\textcolor[rgb]{0.25,0.50,0.50}{##1}}}
\expandafter\def\csname PY@tok@cs\endcsname{\let\PY@it=\textit\def\PY@tc##1{\textcolor[rgb]{0.25,0.50,0.50}{##1}}}

\def\PYZbs{\char`\\}
\def\PYZus{\char`\_}
\def\PYZob{\char`\{}
\def\PYZcb{\char`\}}
\def\PYZca{\char`\^}
\def\PYZam{\char`\&}
\def\PYZlt{\char`\<}
\def\PYZgt{\char`\>}
\def\PYZsh{\char`\#}
\def\PYZpc{\char`\%}
\def\PYZdl{\char`\$}
\def\PYZhy{\char`\-}
\def\PYZsq{\char`\'}
\def\PYZdq{\char`\"}
\def\PYZti{\char`\~}
% for compatibility with earlier versions
\def\PYZat{@}
\def\PYZlb{[}
\def\PYZrb{]}
\makeatother


    % Exact colors from NB
    \definecolor{incolor}{rgb}{0.0, 0.0, 0.5}
    \definecolor{outcolor}{rgb}{0.545, 0.0, 0.0}



    
    % Prevent overflowing lines due to hard-to-break entities
    \sloppy 
    % Setup hyperref package
    \hypersetup{
      breaklinks=true,  % so long urls are correctly broken across lines
      colorlinks=true,
      urlcolor=urlcolor,
      linkcolor=linkcolor,
      citecolor=citecolor,
      }
    % Slightly bigger margins than the latex defaults
    
    \geometry{verbose,tmargin=1in,bmargin=1in,lmargin=1in,rmargin=1in}
    
    

    \begin{document}
    
    
    \maketitle
    
    

    
    \section{Object Detection for Auto
Driving}\label{object-detection-for-auto-driving}

    \begin{Verbatim}[commandchars=\\\{\}]
{\color{incolor}In [{\color{incolor}16}]:} \PY{k+kn}{import} \PY{n+nn}{os}
         \PY{k+kn}{import} \PY{n+nn}{warnings}
         \PY{n}{warnings}\PY{o}{.}\PY{n}{filterwarnings}\PY{p}{(}\PY{l+s+s1}{\PYZsq{}}\PY{l+s+s1}{ignore}\PY{l+s+s1}{\PYZsq{}}\PY{p}{)}
         \PY{k+kn}{from} \PY{n+nn}{matplotlib}\PY{n+nn}{.}\PY{n+nn}{pyplot} \PY{k}{import} \PY{n}{imshow}
         \PY{k+kn}{import} \PY{n+nn}{scipy}\PY{n+nn}{.}\PY{n+nn}{io}
         \PY{k+kn}{import} \PY{n+nn}{scipy}\PY{n+nn}{.}\PY{n+nn}{misc}
         \PY{k+kn}{import} \PY{n+nn}{tensorflow} \PY{k}{as} \PY{n+nn}{tf}
         \PY{k+kn}{from} \PY{n+nn}{keras} \PY{k}{import} \PY{n}{backend} \PY{k}{as} \PY{n}{K}
         \PY{k+kn}{from} \PY{n+nn}{keras}\PY{n+nn}{.}\PY{n+nn}{models} \PY{k}{import} \PY{n}{load\PYZus{}model}
         \PY{k+kn}{from} \PY{n+nn}{yolo\PYZus{}utils} \PY{k}{import} \PY{n}{read\PYZus{}classes}\PY{p}{,} \PY{n}{read\PYZus{}anchors}\PY{p}{,} \PY{n}{generate\PYZus{}colors}\PY{p}{,} \PY{n}{preprocess\PYZus{}image}\PY{p}{,} \PY{n}{draw\PYZus{}boxes}\PY{p}{,} \PY{n}{scale\PYZus{}boxes}
         \PY{k+kn}{from} \PY{n+nn}{yad2k}\PY{n+nn}{.}\PY{n+nn}{models}\PY{n+nn}{.}\PY{n+nn}{keras\PYZus{}yolo} \PY{k}{import} \PY{n}{yolo\PYZus{}head}\PY{p}{,} \PY{n}{yolo\PYZus{}boxes\PYZus{}to\PYZus{}corners}
\end{Verbatim}


    \subsection{1 Problem}\label{problem}

    Dataset is provided by \href{https://www.drive.ai/}{drive.ai}. Images
were gathered from cameras mounted to the front of cars. We want to use
YOLO algorithm to recognize objects in images. Recognized objects are
labelled by a square box. In the notebook, I did following: - F - max -

    \subsubsection{Definition of a box}\label{definition-of-a-box}

\(b_x\) and \(b_y\) define center of box and \(b_h\) and \(b_w\) define
size of box. If there are 80 categories to recognize, I can either
represent the category of object by: - \(i)\) label c as an integer from
1 to 80: 6 elements to represent a box - \(ii)\) one hot vector with
\$c\_\{th\} \$ place as 1 and all others as 0s: 85 elements to represent
a box

    \subsection{2 YOLO}\label{yolo}

    YOLO ("you only look once") requires only one forward propagation pass
through the network to make predictions. Thus it "only looks once" at
the image.

    \subsubsection{2.1 Model details}\label{model-details}

\begin{itemize}
\tightlist
\item
  The \textbf{input} is m images in tensor of shape (m, 608, 608, 3)
\item
  The \textbf{output} is a list of boxes along with the recognized
  classes (m, 19, 19, 5, 85). Each image is cut into 19*19 cells. Each
  cell has five boxes. Each bounding box is represented by 6 numbers
  \((p_c, b_x, b_y, b_h, b_w, c)\) as explained above. If \(c\) is
  expanded into an 80-dimensional vector, each bounding box is then
  represented by 85 numbers.
\end{itemize}

If the center/midpoint of an object falls into a grid cell, that grid
cell is responsible for detecting that object. A cell can have maximum
of 5 objects centered inside.

YOLO architecture: IMAGE (m, 608, 608, 3) -\textgreater{} DEEP CNN
-\textgreater{} ENCODING (m, 19, 19, 5, 85).

    \subsubsection{2.2 - Filtering boxes with class
scores}\label{filtering-boxes-with-class-scores}

Each cell gives 5 boxes. So the model can predict 19x19x5=1805 boxes by
just looking once at the image. So we need - First, only keep boxes with
high class score (more confident about detecting an object) - Second,
only keep one box when several overlapping boxes are detecting the same
object 

    \textbf{yolo\_filter\_boxes( box\_confidence, boxes, box\_class\_probs,
threshold)} will filter boxes:

Step 1: Scores of every class are calculated by \(p_c\) * (\(c_1\),
\(c_2\), ..., \(c_{79}\), \(c_{80}\)) - "box\_confidence" is \(p_c\), a
tensor of shape (19, 19, 5, 1) - "box\_class\_probs" is ( \(c_1\),
\(c_2\), ..., \(c_{79}\), \(c_{80}\)), a tensor of shape (19, 19, 5, 80)
- "boxes" is sizes of all the boxes, containing
\((b_x, b_y, b_h, b_w)\), a tensor of shape (19, 19, 5, 4)

Step 2: In every box, find the index and value of class with max score.
Index is saved as "box\_classes" and value is saved as
"box\_class\_scores". Create a filtering mask based on
"box\_class\_scores" by using "threshold".

Step 3: Apply filtering mask to all boxes and got boxes with scores
higher than threshold. - "scores" -\/- tensor of shape
(number\_selected\_boxes, 1), containing the class probability score for
selected boxes - "boxes" -\/- tensor of shape (number\_selected\_boxes,
4), containing \((b_x, b_y, b_h, b_w)\) coordinates of selected boxes -
"classes" -\/- tensor of shape (number\_selected\_boxes, 1), containing
the index of the class detected by the selected boxes

    \begin{Verbatim}[commandchars=\\\{\}]
{\color{incolor}In [{\color{incolor}17}]:} \PY{k}{def} \PY{n+nf}{yolo\PYZus{}filter\PYZus{}boxes}\PY{p}{(}\PY{n}{box\PYZus{}confidence}\PY{p}{,} \PY{n}{boxes}\PY{p}{,} \PY{n}{box\PYZus{}class\PYZus{}probs}\PY{p}{,} \PY{n}{threshold} \PY{o}{=} \PY{o}{.}\PY{l+m+mi}{6}\PY{p}{)}\PY{p}{:}
             \PY{c+c1}{\PYZsh{} Step 1: Compute box scores.}
             \PY{n}{box\PYZus{}scores} \PY{o}{=} \PY{n}{box\PYZus{}confidence} \PY{o}{*} \PY{n}{box\PYZus{}class\PYZus{}probs}
             
             \PY{c+c1}{\PYZsh{} Step 2: find the index and value of class with max score.}
             \PY{n}{box\PYZus{}classes} \PY{o}{=} \PY{n}{K}\PY{o}{.}\PY{n}{argmax}\PY{p}{(}\PY{n}{box\PYZus{}scores}\PY{p}{,} \PY{n}{axis}\PY{o}{=}\PY{o}{\PYZhy{}}\PY{l+m+mi}{1}\PY{p}{)}
             \PY{n}{box\PYZus{}class\PYZus{}scores} \PY{o}{=} \PY{n}{K}\PY{o}{.}\PY{n}{max}\PY{p}{(}\PY{n}{box\PYZus{}scores}\PY{p}{,} \PY{n}{axis}\PY{o}{=}\PY{o}{\PYZhy{}}\PY{l+m+mi}{1}\PY{p}{,} \PY{n}{keepdims}\PY{o}{=}\PY{k+kc}{False}\PY{p}{)}
             \PY{c+c1}{\PYZsh{} Create a filtering mask based on \PYZdq{}box\PYZus{}class\PYZus{}scores\PYZdq{} by using \PYZdq{}threshold\PYZdq{}. The mask have the}
             \PY{c+c1}{\PYZsh{} same dimension as box\PYZus{}class\PYZus{}scores, and be True for the boxes you want to keep }
             \PY{n}{filtering\PYZus{}mask} \PY{o}{=} \PY{n}{box\PYZus{}class\PYZus{}scores}\PY{o}{\PYZgt{}}\PY{o}{=}\PY{n}{threshold}
             
             \PY{c+c1}{\PYZsh{} Step 3: Apply the mask to scores, boxes and classes, select box with score higher than threshold}
             \PY{n}{scores} \PY{o}{=} \PY{n}{tf}\PY{o}{.}\PY{n}{boolean\PYZus{}mask}\PY{p}{(}\PY{n}{box\PYZus{}class\PYZus{}scores}\PY{p}{,} \PY{n}{filtering\PYZus{}mask}\PY{p}{)}
             \PY{n}{boxes} \PY{o}{=} \PY{n}{tf}\PY{o}{.}\PY{n}{boolean\PYZus{}mask}\PY{p}{(}\PY{n}{boxes}\PY{p}{,} \PY{n}{filtering\PYZus{}mask}\PY{p}{)}
             \PY{n}{classes} \PY{o}{=} \PY{n}{tf}\PY{o}{.}\PY{n}{boolean\PYZus{}mask}\PY{p}{(}\PY{n}{box\PYZus{}classes}\PY{p}{,} \PY{n}{filtering\PYZus{}mask}\PY{p}{)}
             
             \PY{k}{return} \PY{n}{scores}\PY{p}{,} \PY{n}{boxes}\PY{p}{,} \PY{n}{classes}
\end{Verbatim}


    \subsubsection{2.3 - Non-max suppression}\label{non-max-suppression}

After filtering by thresholding over the classes scores, I end up a lot
of overlapping boxes. A second filter for selecting the right boxes is
called non-maximum suppression (NMS). 

    Non-max suppression uses the very important function called
\textbf{"Intersection over Union"}, or IoU. 

    \textbf{iou(box1, box2)} calculates IoU shown above. \textbf{IoU
large-\textgreater{}more overlap-\textgreater{}delete}. (x1, y1, x2, y2)
is upper left and lower right in box.

box1 -\/- first box, list object with coordinates (x1, y1, x2, y2)

box2 -\/- second box, list object with coordinates (x1, y1, x2, y2)

    \begin{Verbatim}[commandchars=\\\{\}]
{\color{incolor}In [{\color{incolor}18}]:} \PY{k}{def} \PY{n+nf}{iou}\PY{p}{(}\PY{n}{box1}\PY{p}{,} \PY{n}{box2}\PY{p}{)}\PY{p}{:}
             
             \PY{c+c1}{\PYZsh{} Calculate its INTER Area.}
             \PY{n}{xi1} \PY{o}{=} \PY{n+nb}{max}\PY{p}{(}\PY{n}{box1}\PY{p}{[}\PY{l+m+mi}{0}\PY{p}{]}\PY{p}{,}\PY{n}{box2}\PY{p}{[}\PY{l+m+mi}{0}\PY{p}{]}\PY{p}{)}
             \PY{n}{yi1} \PY{o}{=} \PY{n+nb}{max}\PY{p}{(}\PY{n}{box1}\PY{p}{[}\PY{l+m+mi}{1}\PY{p}{]}\PY{p}{,}\PY{n}{box2}\PY{p}{[}\PY{l+m+mi}{1}\PY{p}{]}\PY{p}{)}
             \PY{n}{xi2} \PY{o}{=} \PY{n+nb}{min}\PY{p}{(}\PY{n}{box1}\PY{p}{[}\PY{l+m+mi}{2}\PY{p}{]}\PY{p}{,}\PY{n}{box2}\PY{p}{[}\PY{l+m+mi}{2}\PY{p}{]}\PY{p}{)}
             \PY{n}{yi2} \PY{o}{=} \PY{n+nb}{min}\PY{p}{(}\PY{n}{box1}\PY{p}{[}\PY{l+m+mi}{3}\PY{p}{]}\PY{p}{,}\PY{n}{box2}\PY{p}{[}\PY{l+m+mi}{3}\PY{p}{]}\PY{p}{)}
             \PY{n}{inter\PYZus{}area} \PY{o}{=} \PY{p}{(}\PY{n}{xi2}\PY{o}{\PYZhy{}}\PY{n}{xi1}\PY{p}{)}\PY{o}{*}\PY{p}{(}\PY{n}{yi2}\PY{o}{\PYZhy{}}\PY{n}{yi1}\PY{p}{)}
         
             \PY{c+c1}{\PYZsh{} Calculate the Union area by using Formula: Union(A,B) = A + B \PYZhy{} Inter(A,B)}
             \PY{n}{box1\PYZus{}area} \PY{o}{=} \PY{p}{(}\PY{n}{box1}\PY{p}{[}\PY{l+m+mi}{3}\PY{p}{]} \PY{o}{\PYZhy{}} \PY{n}{box1}\PY{p}{[}\PY{l+m+mi}{1}\PY{p}{]}\PY{p}{)} \PY{o}{*} \PY{p}{(}\PY{n}{box1}\PY{p}{[}\PY{l+m+mi}{2}\PY{p}{]} \PY{o}{\PYZhy{}} \PY{n}{box1}\PY{p}{[}\PY{l+m+mi}{0}\PY{p}{]}\PY{p}{)}
             \PY{n}{box2\PYZus{}area} \PY{o}{=} \PY{p}{(}\PY{n}{box2}\PY{p}{[}\PY{l+m+mi}{3}\PY{p}{]} \PY{o}{\PYZhy{}} \PY{n}{box2}\PY{p}{[}\PY{l+m+mi}{1}\PY{p}{]}\PY{p}{)} \PY{o}{*} \PY{p}{(}\PY{n}{box2}\PY{p}{[}\PY{l+m+mi}{2}\PY{p}{]} \PY{o}{\PYZhy{}} \PY{n}{box2}\PY{p}{[}\PY{l+m+mi}{0}\PY{p}{]}\PY{p}{)}
             \PY{n}{union\PYZus{}area} \PY{o}{=} \PY{n}{box1\PYZus{}area}\PY{o}{+}\PY{n}{box2\PYZus{}area}\PY{o}{\PYZhy{}}\PY{n}{inter\PYZus{}area}
             
             \PY{c+c1}{\PYZsh{} compute the IoU}
             \PY{n}{iou} \PY{o}{=} \PY{n}{inter\PYZus{}area}\PY{o}{/}\PY{n}{union\PYZus{}area}
         
             \PY{k}{return} \PY{n}{iou}
\end{Verbatim}


    \textbf{yolo\_non\_max\_suppression(scores, boxes, classes, max\_boxes =
10, iou\_threshold = 0.5)}

Implement non-max suppression. The key steps are: 1. Select the box that
has the highest score. 2. Compute its overlap with all other boxes, and
remove boxes that overlap it more than "iou\_threshold". 3. Go back to
step 1 and iterate until the selected box has lowest score among all
boxes

Arguments:

"scores","boxes","classes" are output of yolo\_filter\_boxes().
"max\_boxes": maximum number of predicted boxes you'd like

Returns:

"scores","boxes","classes" selected boxes after non\_max\_suppression

    \begin{Verbatim}[commandchars=\\\{\}]
{\color{incolor}In [{\color{incolor}19}]:} \PY{k}{def} \PY{n+nf}{yolo\PYZus{}non\PYZus{}max\PYZus{}suppression}\PY{p}{(}\PY{n}{scores}\PY{p}{,} \PY{n}{boxes}\PY{p}{,} \PY{n}{classes}\PY{p}{,} \PY{n}{max\PYZus{}boxes} \PY{o}{=} \PY{l+m+mi}{10}\PY{p}{,} \PY{n}{iou\PYZus{}threshold} \PY{o}{=} \PY{l+m+mf}{0.5}\PY{p}{)}\PY{p}{:}
             
             \PY{c+c1}{\PYZsh{} tensor to be used in tf.image.non\PYZus{}max\PYZus{}suppression()}
             \PY{n}{max\PYZus{}boxes\PYZus{}tensor} \PY{o}{=} \PY{n}{K}\PY{o}{.}\PY{n}{variable}\PY{p}{(}\PY{n}{max\PYZus{}boxes}\PY{p}{,} \PY{n}{dtype}\PY{o}{=}\PY{l+s+s1}{\PYZsq{}}\PY{l+s+s1}{int32}\PY{l+s+s1}{\PYZsq{}}\PY{p}{)}
             
             \PY{c+c1}{\PYZsh{} initialize variable max\PYZus{}boxes\PYZus{}tensor}
             \PY{n}{K}\PY{o}{.}\PY{n}{get\PYZus{}session}\PY{p}{(}\PY{p}{)}\PY{o}{.}\PY{n}{run}\PY{p}{(}\PY{n}{tf}\PY{o}{.}\PY{n}{variables\PYZus{}initializer}\PY{p}{(}\PY{p}{[}\PY{n}{max\PYZus{}boxes\PYZus{}tensor}\PY{p}{]}\PY{p}{)}\PY{p}{)} 
             
             \PY{c+c1}{\PYZsh{} Use tf.image.non\PYZus{}max\PYZus{}suppression() to get the list of indices corresponding to boxes you keep}
             \PY{n}{nms\PYZus{}indices} \PY{o}{=} \PY{n}{tf}\PY{o}{.}\PY{n}{image}\PY{o}{.}\PY{n}{non\PYZus{}max\PYZus{}suppression}\PY{p}{(}\PY{n}{boxes}\PY{p}{,}\PY{n}{scores}\PY{p}{,}\PY{n}{max\PYZus{}boxes}\PY{p}{,} \PY{n}{iou\PYZus{}threshold}\PY{p}{,}\PY{n}{name}\PY{o}{=}\PY{k+kc}{None}\PY{p}{)}
             
             \PY{c+c1}{\PYZsh{} Use K.gather() to select only nms\PYZus{}indices from scores, boxes and classes}
             \PY{n}{scores} \PY{o}{=} \PY{n}{K}\PY{o}{.}\PY{n}{gather}\PY{p}{(}\PY{n}{scores}\PY{p}{,}\PY{n}{nms\PYZus{}indices}\PY{p}{)}
             \PY{n}{boxes} \PY{o}{=} \PY{n}{K}\PY{o}{.}\PY{n}{gather}\PY{p}{(}\PY{n}{boxes}\PY{p}{,}\PY{n}{nms\PYZus{}indices}\PY{p}{)}
             \PY{n}{classes} \PY{o}{=} \PY{n}{K}\PY{o}{.}\PY{n}{gather}\PY{p}{(}\PY{n}{classes}\PY{p}{,}\PY{n}{nms\PYZus{}indices}\PY{p}{)}
             
             \PY{k}{return} \PY{n}{scores}\PY{p}{,} \PY{n}{boxes}\PY{p}{,} \PY{n}{classes}
\end{Verbatim}


    \subsubsection{2.4 Wrapping up these two
filtering}\label{wrapping-up-these-two-filtering}

\textbf{yolo\_eval(yolo\_outputs, image\_shape, max\_boxes,
score\_threshold, iou\_threshold)} use the output of the deep CNN (the
19x19x5x85 dimensional encoding) and filtering through all the boxes
using the functions just implemented.

Arguments:

"yolo\_outputs" -\/- output of the deep CNN model (for image\_shape of
(608, 608, 3)), contains 4 tensors:

\begin{verbatim}
                box_confidence: tensor of shape (None, 19, 19, 5, 1)
                box_xy: tensor of shape (None, 19, 19, 5, 2)
                box_wh: tensor of shape (None, 19, 19, 5, 2)
                box_class_probs: tensor of shape (None, 19, 19, 5, 80)
\end{verbatim}

"image\_shape" -\/- tensor of shape (2,) containing the input shape

Returns:

"scores", "boxes", "classes" are all output of
yolo\_non\_max\_suppression

\textbf{Note:boxes = scale\_boxes(boxes, image\_shape)} rescales the
boxes:

YOLO's network was trained to run on 608x608 images. To test this data
on a different size-\/-for example, 720x1280 images-\/-need to rescale
the boxes.

    \begin{Verbatim}[commandchars=\\\{\}]
{\color{incolor}In [{\color{incolor}21}]:} \PY{k}{def} \PY{n+nf}{yolo\PYZus{}eval}\PY{p}{(}\PY{n}{yolo\PYZus{}outputs}\PY{p}{,} \PY{n}{image\PYZus{}shape} \PY{o}{=} \PY{p}{(}\PY{l+m+mf}{720.}\PY{p}{,} \PY{l+m+mf}{1280.}\PY{p}{)}\PY{p}{,} \PY{n}{max\PYZus{}boxes}\PY{o}{=}\PY{l+m+mi}{10}\PY{p}{,} \PY{n}{score\PYZus{}threshold}\PY{o}{=}\PY{o}{.}\PY{l+m+mi}{6}\PY{p}{,} \PY{n}{iou\PYZus{}threshold}\PY{o}{=}\PY{o}{.}\PY{l+m+mi}{5}\PY{p}{)}\PY{p}{:}
             
             \PY{c+c1}{\PYZsh{} Retrieve outputs of the YOLO model }
             \PY{n}{box\PYZus{}confidence}\PY{p}{,} \PY{n}{box\PYZus{}xy}\PY{p}{,} \PY{n}{box\PYZus{}wh}\PY{p}{,} \PY{n}{box\PYZus{}class\PYZus{}probs} \PY{o}{=} \PY{n}{yolo\PYZus{}outputs}
         
             \PY{c+c1}{\PYZsh{} Convert boxes to be ready for filtering functions }
             \PY{n}{boxes} \PY{o}{=} \PY{n}{yolo\PYZus{}boxes\PYZus{}to\PYZus{}corners}\PY{p}{(}\PY{n}{box\PYZus{}xy}\PY{p}{,} \PY{n}{box\PYZus{}wh}\PY{p}{)}
         
             \PY{c+c1}{\PYZsh{} perform Score\PYZhy{}filtering with a threshold of score\PYZus{}threshold}
             \PY{n}{scores}\PY{p}{,} \PY{n}{boxes}\PY{p}{,} \PY{n}{classes} \PY{o}{=} \PY{n}{yolo\PYZus{}filter\PYZus{}boxes}\PY{p}{(}\PY{n}{box\PYZus{}confidence}\PY{p}{,} \PY{n}{boxes}\PY{p}{,} \PY{n}{box\PYZus{}class\PYZus{}probs}\PY{p}{,} \PY{n}{score\PYZus{}threshold}\PY{p}{)}
             
             \PY{c+c1}{\PYZsh{} Scale boxes back to original image shape}
             \PY{n}{boxes} \PY{o}{=} \PY{n}{scale\PYZus{}boxes}\PY{p}{(}\PY{n}{boxes}\PY{p}{,} \PY{n}{image\PYZus{}shape}\PY{p}{)}
         
             \PY{c+c1}{\PYZsh{} perform Non\PYZhy{}max suppression with a threshold of iou\PYZus{}threshold }
             \PY{n}{scores}\PY{p}{,} \PY{n}{boxes}\PY{p}{,} \PY{n}{classes} \PY{o}{=} \PY{n}{yolo\PYZus{}non\PYZus{}max\PYZus{}suppression}\PY{p}{(}\PY{n}{scores}\PY{p}{,} \PY{n}{boxes}\PY{p}{,} \PY{n}{classes}\PY{p}{,} \PY{n}{max\PYZus{}boxes}\PY{p}{,} \PY{n}{iou\PYZus{}threshold}\PY{p}{)}
         
             \PY{k}{return} \PY{n}{scores}\PY{p}{,} \PY{n}{boxes}\PY{p}{,} \PY{n}{classes}
\end{Verbatim}


    \begin{Verbatim}[commandchars=\\\{\}]
{\color{incolor}In [{\color{incolor}22}]:} \PY{n}{sess} \PY{o}{=} \PY{n}{K}\PY{o}{.}\PY{n}{get\PYZus{}session}\PY{p}{(}\PY{p}{)}
\end{Verbatim}


    \begin{Verbatim}[commandchars=\\\{\}]
{\color{incolor}In [{\color{incolor}23}]:} \PY{n}{class\PYZus{}names} \PY{o}{=} \PY{n}{read\PYZus{}classes}\PY{p}{(}\PY{l+s+s2}{\PYZdq{}}\PY{l+s+s2}{model\PYZus{}data/coco\PYZus{}classes.txt}\PY{l+s+s2}{\PYZdq{}}\PY{p}{)}
         \PY{n}{anchors} \PY{o}{=} \PY{n}{read\PYZus{}anchors}\PY{p}{(}\PY{l+s+s2}{\PYZdq{}}\PY{l+s+s2}{model\PYZus{}data/yolo\PYZus{}anchors.txt}\PY{l+s+s2}{\PYZdq{}}\PY{p}{)}
         \PY{n}{image\PYZus{}shape} \PY{o}{=} \PY{p}{(}\PY{l+m+mf}{720.}\PY{p}{,} \PY{l+m+mf}{1280.}\PY{p}{)}  
\end{Verbatim}


    \begin{Verbatim}[commandchars=\\\{\}]
{\color{incolor}In [{\color{incolor}24}]:} \PY{n}{yolo\PYZus{}model} \PY{o}{=} \PY{n}{load\PYZus{}model}\PY{p}{(}\PY{l+s+s2}{\PYZdq{}}\PY{l+s+s2}{model\PYZus{}data/yolo.h5}\PY{l+s+s2}{\PYZdq{}}\PY{p}{)}
\end{Verbatim}


    \begin{Verbatim}[commandchars=\\\{\}]
{\color{incolor}In [{\color{incolor}25}]:} \PY{n}{yolo\PYZus{}outputs} \PY{o}{=} \PY{n}{yolo\PYZus{}head}\PY{p}{(}\PY{n}{yolo\PYZus{}model}\PY{o}{.}\PY{n}{output}\PY{p}{,} \PY{n}{anchors}\PY{p}{,} \PY{n+nb}{len}\PY{p}{(}\PY{n}{class\PYZus{}names}\PY{p}{)}\PY{p}{)}
\end{Verbatim}


    \begin{Verbatim}[commandchars=\\\{\}]
{\color{incolor}In [{\color{incolor}26}]:} \PY{n}{scores}\PY{p}{,} \PY{n}{boxes}\PY{p}{,} \PY{n}{classes} \PY{o}{=} \PY{n}{yolo\PYZus{}eval}\PY{p}{(}\PY{n}{yolo\PYZus{}outputs}\PY{p}{,} \PY{n}{image\PYZus{}shape}\PY{p}{)}
\end{Verbatim}


    \begin{Verbatim}[commandchars=\\\{\}]
{\color{incolor}In [{\color{incolor}27}]:} \PY{k}{def} \PY{n+nf}{predict}\PY{p}{(}\PY{n}{sess}\PY{p}{,} \PY{n}{image\PYZus{}file}\PY{p}{)}\PY{p}{:}
             \PY{l+s+sd}{\PYZdq{}\PYZdq{}\PYZdq{}}
         \PY{l+s+sd}{    Runs the graph stored in \PYZdq{}sess\PYZdq{} to predict boxes for \PYZdq{}image\PYZus{}file\PYZdq{}. Prints and plots the preditions.}
         \PY{l+s+sd}{    }
         \PY{l+s+sd}{    Arguments:}
         \PY{l+s+sd}{    sess \PYZhy{}\PYZhy{} your tensorflow/Keras session containing the YOLO graph}
         \PY{l+s+sd}{    image\PYZus{}file \PYZhy{}\PYZhy{} name of an image stored in the \PYZdq{}images\PYZdq{} folder.}
         \PY{l+s+sd}{    }
         \PY{l+s+sd}{    Returns:}
         \PY{l+s+sd}{    out\PYZus{}scores \PYZhy{}\PYZhy{} tensor of shape (None, ), scores of the predicted boxes}
         \PY{l+s+sd}{    out\PYZus{}boxes \PYZhy{}\PYZhy{} tensor of shape (None, 4), coordinates of the predicted boxes}
         \PY{l+s+sd}{    out\PYZus{}classes \PYZhy{}\PYZhy{} tensor of shape (None, ), class index of the predicted boxes}
         \PY{l+s+sd}{    }
         \PY{l+s+sd}{    Note: \PYZdq{}None\PYZdq{} actually represents the number of predicted boxes, it varies between 0 and max\PYZus{}boxes. }
         \PY{l+s+sd}{    \PYZdq{}\PYZdq{}\PYZdq{}}
         
             \PY{c+c1}{\PYZsh{} Preprocess your image}
             \PY{n}{image}\PY{p}{,} \PY{n}{image\PYZus{}data} \PY{o}{=} \PY{n}{preprocess\PYZus{}image}\PY{p}{(}\PY{l+s+s2}{\PYZdq{}}\PY{l+s+s2}{images/}\PY{l+s+s2}{\PYZdq{}} \PY{o}{+} \PY{n}{image\PYZus{}file}\PY{p}{,} \PY{n}{model\PYZus{}image\PYZus{}size} \PY{o}{=} \PY{p}{(}\PY{l+m+mi}{608}\PY{p}{,} \PY{l+m+mi}{608}\PY{p}{)}\PY{p}{)}
         
             \PY{c+c1}{\PYZsh{} Run the session with the correct tensors and choose the correct placeholders in the feed\PYZus{}dict.}
             \PY{c+c1}{\PYZsh{} You\PYZsq{}ll need to use feed\PYZus{}dict=\PYZob{}yolo\PYZus{}model.input: ... , K.learning\PYZus{}phase(): 0\PYZcb{})}
             \PY{c+c1}{\PYZsh{}\PYZsh{}\PYZsh{} START CODE HERE \PYZsh{}\PYZsh{}\PYZsh{} (≈ 1 line)}
             \PY{n}{out\PYZus{}scores}\PY{p}{,} \PY{n}{out\PYZus{}boxes}\PY{p}{,} \PY{n}{out\PYZus{}classes} \PY{o}{=} \PY{n}{sess}\PY{o}{.}\PY{n}{run}\PY{p}{(}\PY{n}{yolo\PYZus{}eval}\PY{p}{(}\PY{n}{yolo\PYZus{}outputs}\PY{p}{,} \PY{n}{image\PYZus{}shape}\PY{p}{)}\PY{p}{,}\PY{n}{feed\PYZus{}dict}\PY{o}{=}\PY{p}{\PYZob{}}\PY{n}{yolo\PYZus{}model}\PY{o}{.}\PY{n}{input}\PY{p}{:} \PY{n}{image\PYZus{}data}\PY{p}{,} \PY{n}{K}\PY{o}{.}\PY{n}{learning\PYZus{}phase}\PY{p}{(}\PY{p}{)}\PY{p}{:} \PY{l+m+mi}{0}\PY{p}{\PYZcb{}}\PY{p}{)}
             \PY{c+c1}{\PYZsh{}\PYZsh{}\PYZsh{} END CODE HERE \PYZsh{}\PYZsh{}\PYZsh{}}
         
             \PY{c+c1}{\PYZsh{} Print predictions info}
             \PY{n+nb}{print}\PY{p}{(}\PY{l+s+s1}{\PYZsq{}}\PY{l+s+s1}{Found }\PY{l+s+si}{\PYZob{}\PYZcb{}}\PY{l+s+s1}{ boxes for }\PY{l+s+si}{\PYZob{}\PYZcb{}}\PY{l+s+s1}{\PYZsq{}}\PY{o}{.}\PY{n}{format}\PY{p}{(}\PY{n+nb}{len}\PY{p}{(}\PY{n}{out\PYZus{}boxes}\PY{p}{)}\PY{p}{,} \PY{n}{image\PYZus{}file}\PY{p}{)}\PY{p}{)}
             \PY{c+c1}{\PYZsh{} Generate colors for drawing bounding boxes.}
             \PY{n}{colors} \PY{o}{=} \PY{n}{generate\PYZus{}colors}\PY{p}{(}\PY{n}{class\PYZus{}names}\PY{p}{)}
             \PY{c+c1}{\PYZsh{} Draw bounding boxes on the image file}
             \PY{n}{draw\PYZus{}boxes}\PY{p}{(}\PY{n}{image}\PY{p}{,} \PY{n}{out\PYZus{}scores}\PY{p}{,} \PY{n}{out\PYZus{}boxes}\PY{p}{,} \PY{n}{out\PYZus{}classes}\PY{p}{,} \PY{n}{class\PYZus{}names}\PY{p}{,} \PY{n}{colors}\PY{p}{)}
             \PY{c+c1}{\PYZsh{} Save the predicted bounding box on the image}
             \PY{n}{image}\PY{o}{.}\PY{n}{save}\PY{p}{(}\PY{n}{os}\PY{o}{.}\PY{n}{path}\PY{o}{.}\PY{n}{join}\PY{p}{(}\PY{l+s+s2}{\PYZdq{}}\PY{l+s+s2}{out}\PY{l+s+s2}{\PYZdq{}}\PY{p}{,} \PY{n}{image\PYZus{}file}\PY{p}{)}\PY{p}{,} \PY{n}{quality}\PY{o}{=}\PY{l+m+mi}{90}\PY{p}{)}
             \PY{c+c1}{\PYZsh{} Display the results in the notebook}
             \PY{n}{output\PYZus{}image} \PY{o}{=} \PY{n}{scipy}\PY{o}{.}\PY{n}{misc}\PY{o}{.}\PY{n}{imread}\PY{p}{(}\PY{n}{os}\PY{o}{.}\PY{n}{path}\PY{o}{.}\PY{n}{join}\PY{p}{(}\PY{l+s+s2}{\PYZdq{}}\PY{l+s+s2}{out}\PY{l+s+s2}{\PYZdq{}}\PY{p}{,} \PY{n}{image\PYZus{}file}\PY{p}{)}\PY{p}{)}
             \PY{n}{imshow}\PY{p}{(}\PY{n}{output\PYZus{}image}\PY{p}{)}
             
             \PY{k}{return} \PY{n}{out\PYZus{}scores}\PY{p}{,} \PY{n}{out\PYZus{}boxes}\PY{p}{,} \PY{n}{out\PYZus{}classes}
\end{Verbatim}


    \begin{Verbatim}[commandchars=\\\{\}]
{\color{incolor}In [{\color{incolor}28}]:} \PY{n}{out\PYZus{}scores}\PY{p}{,} \PY{n}{out\PYZus{}boxes}\PY{p}{,} \PY{n}{out\PYZus{}classes} \PY{o}{=} \PY{n}{predict}\PY{p}{(}\PY{n}{sess}\PY{p}{,} \PY{l+s+s2}{\PYZdq{}}\PY{l+s+s2}{test.jpg}\PY{l+s+s2}{\PYZdq{}}\PY{p}{)}
\end{Verbatim}


    \begin{Verbatim}[commandchars=\\\{\}]
Found 7 boxes for test.jpg
car 0.60 (925, 285) (1045, 374)
car 0.66 (706, 279) (786, 350)
bus 0.67 (5, 266) (220, 407)
car 0.70 (947, 324) (1280, 705)
car 0.74 (159, 303) (346, 440)
car 0.80 (761, 282) (942, 412)
car 0.89 (367, 300) (745, 648)

    \end{Verbatim}

    \begin{center}
    \adjustimage{max size={0.9\linewidth}{0.9\paperheight}}{output_25_1.png}
    \end{center}
    { \hspace*{\fill} \\}
    

    % Add a bibliography block to the postdoc
    
    
    
    \end{document}
